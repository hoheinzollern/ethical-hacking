\documentclass{scrbook}
\author{Alessandro Bruni}
\title{The Hitchiker's Guide to Ethical Hacking}


\begin{document}
\begin{center}
  {\Huge The Hitchiker's Guide to Ethical Hacking}\\[5em]

  {\huge Don't Panic!}
\end{center}
\newpage
\maketitle

\chapter*{Preface}

\section*{Take your towel, brace yourself!}

Well well, you decided to take the ``Ethical Hacking'' course at the
IT-University of Copenhagen.  Perhaps, you've done so because it is
slightly cheaper than taking a professional ethical hacker
certification.  Or perhaps, just because you were already doing your
masters at ITU and were curious to find out what the fuss was all
about.  You noticed the reassuring message ``Don't Panic'' written in
big letters on the cover of the course handbook and decided that they
must be a trustworthy source of information.  Congratulations!

No, the Earth is not going to be destroyed by an alien species to make
space for an intergalactical highway, not just yet.  Nor are you going
to lose a roof over your head.  But you might, inadvertently or not,
commit a crime that could put you in jail.  Please don't!  We don't
want you in jail.  Jail = bad.  So here is our promise to you: we will
try very hard to help you commit your crime (hack) and get away with
it (with immunity).  Ultimately though, the responsibility is on you,
so if you smell something fishy, or feel like you are walking on
eggshells: stop what you're doing. breathe. talk to us. we will figure
it out.  You are reading this handbook to prepare yourself and prevent
such situations from happening, but you never know\footnote{Nobody
  went to jail in previous editions of this course, and I'd like to
  continue this happy tradition. Thank you!}\ldots

\section*{Follow the white rabbit}

I promise, I will stop with humorous remarks and literature
references, but I am not done just yet.

Hacking is an art.  The amount of knowledge to learn in any specific
topic that we cover in this course is immense.  This handbook is by
definition incomplete.  We will cover a range of topics that will
elevate your hacking skills, from n00b scr1pt k1dd13 2 l33t h4ck3r.
But like painting or any other art form, you do not learn to paint by
reading about how to paint, let alone by reading an incomplete
handbook.  You learn by trying, failing and trying again.  Hopefully
succeeding, eventually.  You will go down a few rabbit holes, and I
hope you will enjoy the learning process, even more than the result.
Embrace frustration.  Do not give up.

Through all this trial and error, you will need to remind yourself of
the bird's eye view.  When you are down in the trenches, trying to
discover a bad flow of information, never forget \emph{why} you are
doing this.  And when (if) you find your smoking gun, always remind
yourself to go back and evaluate the consequences of what you just
found out.  Do not stop at the smoking gun.  Follow the smoke.

\section*{Structure of this course}

This course (and this handbook) is designed to let you:
\begin{enumerate}
\item stay out of jail (first and most importantly!)
\item explore some advanced hacking technique (of your choice)
\item report your findings and get those vulnerabilities fixed
  (hopefully also earn some money in the process, if you choose your
  target wisely)
\end{enumerate}

We will start by introducing the relevant Danish legislation for
hacking, present the coordinated vulnerability disclosure process and
bug bounty programmes in Lecture 1; we will then show you
methodological frameworks for analyzing the security of your target in
Lecture 2.  After this, the fluff is over, and we will switch gears to
dive deep into technical material.  Lecture 3 will give you a refresh
on assembly code and binary reverse engineering.  Lecture 4 will show
you how to automate vulnerability search through symbolic execution.
Lecture 5 will look at deserialization attacks.  Lecture 6 will cover
how to do automated vulnerability scanning with lightweight static
analysis.  Lecture 7 will present fuzzers to randomly search for
vulnerabilities.  Lecture 8 will cover hardware hacking.  Lecture 9
will show you common pitfalls in cryptography.  Lecture 10 will
present forensics and antiforensics.  Lecture 11 will show you
phishing / social engineering techniques.

While we show you all this, you will have to find your target and set
up a coordinated vulnerability disclosure agreement (VDP) with the
appropriate parties.  As you progress through the course, think of
your target and what you are going to do to hack them.  Think well,
what techniques you want to try in your project and dive deep into
them.  Then it will be your turn to shine!  The last 4 weeks of the
course are dedicated to helping you succeed with your project.  Then
finally, will have to report on your experiments, the VDP that you
agreed with the relevant parties, and your findings overall.  In order
to pass the exam, you do not need to discover the hack of the century,
but you will need to convince us that you explored your target with
good enough depth, have understood the security implications of your
hacks, and show that you have remained as legal as possible.

From now on, you get my blessing: \emph{go hack yourself}!

\tableofcontents

\chapter{Legality of hacking and responsible disclosure}

\chapter{Securing your system from a bird's eye view}

\chapter{Binary reverse engineering}

\chapter{Symbolic execution}

\chapter{Static analysis with CodeQL}

\chapter{Deserialization attacks}

\chapter{Fuzzers and randomized vulnerability search}

\chapter{Hardware hacking}

\chapter{Common pitfalls in cryptography}

\chapter{Forensics and antiforensics}

\chapter{Phishing and social engineering}

\end{document}
